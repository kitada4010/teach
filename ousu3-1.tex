\documentclass{jsarticle}
\usepackage{amsmath}

\makeatletter

\def\@thesis{タイトル}
\def\id#1{\def\@id{#1}}
\def\department#1{\def\@department{#1}}

\def\@maketitle{
  \begin{center}
  {\huge \@thesis \par} %修士論文と記載される部分
  \vspace{10mm}
  {\LARGE\bf \@title \par}% 論文のタイトル部分
  \vspace{10mm}
  {\Large \@date\par}% 提出年月日部分
  \vspace{20mm}
  {\Large \@department \par}% 所属部分
  {\Large 学籍番号 \@id \par}% 学籍番号部分
  %{\Large メールアドレス \@email \par}
  \vspace{10mm}
  {\large \@author}% 氏名
  \end{center}
  \par\vskip 1.5em
}

\makeatother

\title{レポート課題 担当:藤澤先生}
\date{\today}
\department{学科コース名など}
\id{学籍番号}
%¥email{v021de@yamaguchi-u.ac.jp}
\author{名前}

\begin{document}
\maketitle
\center
\thanks{E-mailアドレス:}
\newpage


\begin{enumerate}
  \renewcommand{\labelenumii}{\[\arabic{enumii}\]}
\item
  
  \begin{enumerate}
    \renewcommand{\labelenumii}{(\arabic{enumii})}
  \item
    \begin{eqnarray*}
      u &=& \frac{i}{2+3i} \mbox{とおくと}\\
      u &=& \frac{i(2-3i)}{(2+3i)(2-3i)}\\
      &=& \frac{2i+3}{4+9}\\
      &=& \frac{1}{13}(3+2i)\\
      &&\mbox{したがって、複素共役は、}\\
      u^* &=& \frac{1}{13}(3-2i)
    \end{eqnarray*}
  \item
  \end{enumerate}
\end{enumerate}

\end{document}
